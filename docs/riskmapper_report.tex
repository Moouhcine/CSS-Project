\documentclass[11pt]{article}
\usepackage[a4paper,margin=1in]{geometry}
\usepackage{hyperref}
\usepackage{booktabs}
\usepackage{enumitem}
\usepackage{listings}
\usepackage{xcolor}

\lstset{
  basicstyle=\ttfamily\small,
  breaklines=true,
  frame=single
}

\title{\textbf{RiskMapper: Risk and Attack Surface Mapping Tool}\\
\large CVSS-Based Vulnerability Prioritization (Chapter 6: Scanning, CVSS)}
\author{Student Name \quad | \quad CSS Module}
\date{\today}

\begin{document}
\maketitle

\begin{abstract}
RiskMapper is a desktop application built with Python and Flet that translates scanning and
vulnerability data into a standardized risk assessment using the CVSS v3.1 Base Score. The tool
supports manual metric input and bulk importing from CSV/JSON, classifies findings by severity,
and maps vulnerabilities to an attack surface inventory of assets and exposed services to support
remediation prioritization.
\end{abstract}

\section{Objectives}
The project aims to:
\begin{itemize}[leftmargin=*]
  \item Implement CVSS v3.1 Base Score computation (AV, AC, PR, UI, S, C, I, A).
  \item Display both the final score and severity label for prioritization.
  \item Import multiple findings (CSV/JSON) produced by scanners or simplified exports.
  \item Model attack surface: assets, tags (e.g., internet-facing), and exposed services/ports.
  \item Provide asset-centric risk summaries (max/avg score and severity counts).
\end{itemize}

\section{Background: Scanning and CVSS}
Vulnerability scanning can generate hundreds of findings. Not all require immediate action.
CVSS is a standardized method to assess severity and prioritize remediation. RiskMapper focuses on
the \textbf{Base Score} (most commonly used for initial triage) which combines:
\begin{itemize}[leftmargin=*]
  \item \textbf{Exploitability}: AV, AC, PR, UI
  \item \textbf{Impact}: Confidentiality (C), Integrity (I), Availability (A), with Scope (S)
\end{itemize}

\section{System Overview}
\subsection{Technology Choice (Flet)}
Flet enables a modern GUI using Python, without building a web frontend manually. The application
runs locally and is suitable for classroom demonstration.

\subsection{High-level Modules}
\begin{itemize}[leftmargin=*]
  \item \texttt{cvss.py}: CVSS v3.1 Base Score logic and severity classification.
  \item \texttt{parser.py}: CSV/JSON parsing into finding records.
  \item \texttt{storage.py}: in-memory store for assets and findings (educational scope).
  \item \texttt{main.py}: Flet UI with tabs: Dashboard, Calculator, Import, Assets.
\end{itemize}

\section{CVSS v3.1 Base Score Implementation}
\subsection{Supported Metrics}
\begin{center}
\begin{tabular}{ll}
\toprule
Metric & Values \\
\midrule
AV & N, A, L, P \\
AC & L, H \\
PR & N, L, H (depends on Scope) \\
UI & N, R \\
S  & U, C \\
C/I/A & H, L, N \\
\bottomrule
\end{tabular}
\end{center}

\subsection{Computation Summary}
RiskMapper implements the official CVSS v3.1 Base formula:
\begin{itemize}[leftmargin=*]
  \item Exploitability = $8.22 \times AV \times AC \times PR \times UI$
  \item ImpactSubScore = $1 - (1-C)(1-I)(1-A)$
  \item Scope changes impact calculation and scales the final score by 1.08 when Scope is Changed.
  \item Final score is capped at 10.0 and \textbf{rounded up} to one decimal place.
\end{itemize}

\subsection{Severity Levels}
\begin{itemize}[leftmargin=*]
  \item None: 0.0
  \item Low: 0.1--3.9
  \item Medium: 4.0--6.9
  \item High: 7.0--8.9
  \item Critical: 9.0--10.0
\end{itemize}

\section{Attack Surface Mapping}
Attack surface is represented as:
\begin{itemize}[leftmargin=*]
  \item \textbf{Assets}: hostnames, IPs, or applications.
  \item \textbf{Tags}: context such as \textit{internet-facing}, \textit{production}, or \textit{critical}.
  \item \textbf{Services}: exposed ports/protocols (e.g., 443/https, 22/ssh).
\end{itemize}
Findings are linked to assets by matching the \texttt{asset name} string. The Assets page displays
risk summaries (max/avg CVSS) and severity counts, enabling asset-centric prioritization.

\section{Importing Scan Data}
RiskMapper supports:
\begin{itemize}[leftmargin=*]
  \item CSV format with header: \texttt{asset,title,AV,AC,PR,UI,S,C,I,A}
  \item JSON list of objects with the same keys.
\end{itemize}
Rows with invalid or missing metric codes are skipped during import to keep results consistent.

\section{User Interface (Tabs)}
\begin{itemize}[leftmargin=*]
  \item \textbf{Dashboard}: severity counters and latest findings.
  \item \textbf{Calculator}: dropdown metric selection, score + severity output, save finding.
  \item \textbf{Import}: paste or load CSV/JSON and import findings in bulk.
  \item \textbf{Assets}: create assets and view risk per asset + delete operations.
\end{itemize}

\section{Conclusion and Future Work}
RiskMapper demonstrates a complete workflow from vulnerability metrics to actionable prioritization:
CVSS scoring, severity classification, and asset-context risk mapping. Future improvements:
persistent database (SQLite), authentication (multi-user), vector string export, and richer scanner imports.

\end{document}
